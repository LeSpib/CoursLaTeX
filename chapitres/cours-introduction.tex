%%%%%%%%%%%%%%%%%%%%%%%%%%%%%%%%%%%%%%%%
%%%%%  Introduction
%%%%%%%%%%%%%%%%%%%%%%%%%%%%%%%%%%%%%%%%

\chapter{Avant de commencer}
\mtcaddchapter

{\bf \LARGE Hélas} pour vous, le rédacteur de ce qui suit avait envie de se manifester et de vous raconter \og sa vie, son \oe uvre \fg{} dans un subtil texte introductif. Cependant, pour des raisons manifestes de budget papier, votre serviteur a dû se limiter à la seule genèse de ce document. 

Lors de la première année de cette \og conférence \fg{} sur \LaTeX, un élève avait en effet suggéré qu'un polycopié compléterait le cours à merveille. Ce qui a donné ce que vous avez sous les yeux. Toutefois, l'idée à la base de ce document ne consiste pas à faire un parfait polycopié de cours de \LaTeX. Oh que non ! \`{A} cela deux raisons :

\begin{itemize}
\item sur l'Internet et en librairie existent d'excellents documents sur le sujet ; 
\item de façon plus pragmatique, votre rédacteur honni doit sacrifier son temps sur l'autel du travail et du roupillon crapuleux qui s'ensuit... ce qui lui laisse, hélas, fort peu de temps pour faire le polycopié ultime en 14 volumes. \\
\end{itemize}

L'idée qui sous-tend ce document est toute autre : vous disposez ici d'un \emph{exemple}, structuré en deux parties. La première partie revient à ce document final restituant nos choix de mise en forme. Le seconde partie est tout simplement le code source du document. Ce code utilise la plupart des notions et commandes présentées en y ajoutant à quelques reprises de légères subtilités pour les plus curieux d'entre vous. 

Par ailleurs, le peu d'exhaustivité de ce document est compensé à chaque fois que possible par des renvois à des documentations de référence (tel \paquet{minitoc}), des \href{http://fr.wikipedia.org/wiki/Hyperlien}{liens hypertexte}, de la bibliographie ou même des renvois à des pages de trois ouvrages de référence sous la forme suivante :
\incise{32--34,487--501}{24--47}{40--41}
Ces titres --- respectivement de Bernard \textsc{Desgraupes}\cite{desg}, Michel \textsc{Goossens}\cite{gomi} et Christian \textsc{Rolland}\cite{roll} --- sont référencés en bibliographie, en compagnie d'autres ouvrages traitant de \LaTeX, de typographie, de ponctuation \cite{coli} et même de couleur \cite{past} !

Bien évidemment, ce qui suit doit beaucoup aux nombreuses informations, aides et échanges trouvés sur l'Internet. Un grand merci aux passionnés de \LaTeX\ cachés derrière tout cela !